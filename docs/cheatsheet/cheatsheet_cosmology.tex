\usepackage{ifthen}
\usepackage[landscape]{geometry}

% To make this come out properly in landscape mode, do one of the following
% 1.
%  pdflatex latexsheet.tex
%
% 2.
%  latex latexsheet.tex
%  dvips -P pdf  -t landscape latexsheet.dvi
%  ps2pdf latexsheet.ps


% If you're reading this, be prepared for confusion.  Making this was
% a learning experience for me, and it shows.  Much of the placement
% was hacked in; if you make it better, let me know...


% 2008-04
% Changed page margin code to use the geometry package. Also added code for
% conditional page margins, depending on paper size. Thanks to Uwe Ziegenhagen
% for the suggestions.

% 2006-08
% Made changes based on suggestions from Gene Cooperman. <gene at ccs.neu.edu>


% To Do:
% \listoffigures \listoftables
% \setcounter{secnumdepth}{0}


% This sets page margins to .5 inch if using letter paper, and to 1cm
% if using A4 paper. (This probably isn't strictly necessary.)
% If using another size paper, use default 1cm margins.
\ifthenelse{\lengthtest { \paperwidth = 11in}}
	{ \geometry{top=.5in,left=.5in,right=.5in,bottom=.5in} }
	{\ifthenelse{ \lengthtest{ \paperwidth = 297mm}}
		{\geometry{top=1cm,left=1cm,right=1cm,bottom=1cm} }
		{\geometry{top=1cm,left=1cm,right=1cm,bottom=1cm} }
	}

% Turn off header and footer
\pagestyle{empty}


% Redefine section commands to use less space
\makeatletter
\renewcommand{\section}{\@startsection{section}{1}{0mm}%
                                {-1ex plus -.5ex minus -.2ex}%
                                {0.5ex plus .2ex}%x
                                {\normalfont\large\bfseries}}
\renewcommand{\subsection}{\@startsection{subsection}{2}{0mm}%
                                {-1explus -.5ex minus -.2ex}%
                                {0.5ex plus .2ex}%
                                {\normalfont\normalsize\bfseries}}
\renewcommand{\subsubsection}{\@startsection{subsubsection}{3}{0mm}%
                                {-1ex plus -.5ex minus -.2ex}%
                                {1ex plus .2ex}%
                                {\normalfont\small\bfseries}}
\makeatother

% Define BibTeX command
\def\BibTeX{{\rm B\kern-.05em{\sc i\kern-.025em b}\kern-.08em
    T\kern-.1667em\lower.7ex\hbox{E}\kern-.125emX}}

% Don't print section numbers
\setcounter{secnumdepth}{0}


\setlength{\parindent}{0pt}
\setlength{\parskip}{0pt plus 0.5ex}


% -----------------------------------------------------------------------
   \newcommand{\tab}{\hspace*{2em}}

\begin{document}

\raggedright
\footnotesize
\begin{multicols}{3}


% multicol parameters
% These lengths are set only within the two main columns
%\setlength{\columnseprule}{0.25pt}
\setlength{\premulticols}{1pt}
\setlength{\postmulticols}{1pt}
\setlength{\multicolsep}{1pt}
\setlength{\columnsep}{2pt}

\begin{center}
     \Large{\textbf{Cosmology CheatSheet \\ astropy.org}} \\
\end{center}
\section{Partition}
\verb!Path: from astropy.cosmology!\\
\subsection{Methods}
\verb!z_at_value(func,fval,zmin,zmax,ztol,maxfun)! \\
This finds the redshift at which one of the cosmology functions or method is equal to a known value.

\section{FLRW}
Path \verb!astropy.cosmology.core.Cosmology!\\

\subsection{Methods}
\verb!H(z)! \\
Returns the Hubble parameter (km/s/Mpc) at redshift z

\verb!Ob(z)!\\
Returns the density of baryonic matter relative to the critical density at each redshift.

\verb!Ode(z)!
Returns the density of non-relativistic matter relative to the critical density at each redshift.

\verb!Odm(z)!\\
Returns The density of non-relativistic dark matter relative to the critical density at each redshift.

\verb!Ogamma(Z)!\\
Returns the number of all possible permutations.

\verb!Ok(z)!\\
Returns the equivalent density parameter for curvature at each redshift.

\verb!Om(z)!\\
Returns the density of non-relativistic matter relative to the critical density at each redshift.

\verb!Onu(Z)!\\
Returns the energy density of neutrinos relative to the critical density at each redshift.

\verb!Tcmb(Z)! \\
Return the CMB temperature at redshift z

\verb!Tnu(Z)!\\
The temperature of the cosmic neutrino background in K.

\verb!abs_distance_integrand(Z)!\\
The integrand for the absorption distance

\verb!absorption_distance(Z)!\\
Returns the Absorption distance (dimensionless) at each input redshift.

\verb!age(Z)!\\
The age of the universe in Gyr at each input redshift.

\verb!angular_diameter_distance(z)!\\
Returns Angular diameter distance in Mpc at each input redshift.

\verb!angular_diameter_distance_z1z2(z1,z2)!\\
Returns The angular diameter distance between each input redshift pair.

\verb!arcsec_per_kpc_comoving(z)!\\
The angular separation in arcsec corresponding to a comoving kpc at each input redshift.

\verb!clone(**kwargs)!\\
Returns a copy of this object, potentially with some changes.

\verb!comoving_distance(Z)!\\
Returns comoving distance in Mpc to each input redshift
.
\verb!comoving_transverse_distance(z)!\\
Returns comoving transverse distance in Mpc at each input redshift.

\verb!comoving_volume(z)!\\
Returns Comoving volume in cubic Mpc at redshift 

\verb!critical_density(z)!\\
Critical density in grams per cubic cm at redshift z.

\verb!de_density_scale(z)!\\
Evaluates the redshift dependence of the dark energy density.

\verb!differential_comoving_volume(z)!\\
Differential comoving volume per redshift per steradian at each input redshift.

\verb!distmod(z)!\\
Distance modulus at each input redshift, in magnitudes.

 \verb!efunc(z)!\\
 Function used to calculate H(z), the Hubble parameter.

\verb!inv_efunc(z)!\\
The redshift scaling of the inverse Hubble constant.

\verb!kpc_comoving_per_arcmin(z)!\\
The distance in comoving kpc corresponding to an arcmin at each input redshift.

\verb!kpc_proper_per_arcmin(z)!\\
The distance in proper kpc corresponding to an arcmin at each input redshift.

\verb!lookback_distance(z)!\\
Lookback distance in Mpc

\verb!lookback_time(z)!\\
Lookback time in Gyr to each input redshift.

\verb!lookback_time_integrand(z)!\\
The integrand for the lookback time

\verb!lookback_time_integrand(z)!\\
The integrand for the lookback time

\verb!luminosity_distance(z)!\\
Luminosity distance in Mpc at each input redshift.

\verb!nu_relative_density(z)!\\
The neutrino density scaling factor relative to the density in photons at each redshift

\verb!scale_factor(z)!\\
Scale factor at each input redshift.

\verb!w(z)!\\
The dark energy equation of state


\rule{0.3\linewidth}{0.25pt}
\scriptsize\\
http://www.astropy.org


\end{multicols}
\end{document}
